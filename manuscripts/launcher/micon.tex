% 2023本郷祭 マイコン部 部誌 「マイコンHONGOMagazine」 TeXテンプレート

% =環境設定=

\documentclass[b5paper,11pt]{jsarticle}

% 余白
\usepackage[paper=b5j,truedimen,margin=15truemm]{geometry}

% 数式
\usepackage{amsmath,amsfonts}
\usepackage{bm}

% 画像
\usepackage[dvipdfmx]{graphicx}

% 段組
\usepackage{multicol}
\setlength{\columnseprule}{0.5pt}

% ページ番号を削除
\pagestyle{empty}

% ソースコード環境
\usepackage{listings,jvlisting}
\lstset{
  basicstyle={\ttfamily},
  identifierstyle={\small},
  commentstyle={\smallitshape},
  keywordstyle={\small\bfseries},
  ndkeywordstyle={\small},
  stringstyle={\small\ttfamily},
  frame={tb},
  breaklines=true,
  columns=[l]{fullflexible},
  numbers=left,
  xrightmargin=0zw,
  xleftmargin=3zw,
  numberstyle={\scriptsize},
  stepnumber=1,
  numbersep=1zw,
  lineskip=-0.5ex
}

% 書きたい内容に合わせてお好きなパッケージを導入していただいて構いませんが、外部パッケージは提出時に必ず合わせて提出してください。
% また、そのパッケージを使用している部分には必ずコメントをしてください。

% =環境設定ここまで=

\begin{document}

% =タイトル=
\title{ゲームランチャー作成日記}
\author{wattz麻呂}
\date{\today}
\maketitle
% 初めのページのページ番号を削除(\maketitleの影響を回避)
\thispagestyle{empty}

% ここではページごとに段組しています。大きく画像を表示したいときなどは以下を \begin{multicols}{3} 、\end{multicols}のように変更すると文章が終わった直後から空白になります
\begin{multicols*}{3}
  
% 以下本文
こんにちは。(マイコン部の挨拶)再びwattz麻呂です。\\
前回のエミュレーションの話とは打って変わって、文化祭のゲーム詰め合わせCDのランチャーを作った話についてです。\\
UnityでChatGPTの力を借りながら作成したので、その点だけ注意していただければ幸いです。\\
\begin{index}{1}
事の発端
\end{index}
\section{事の発端}
きっかけは8/11、私はモニターマンのポスターを作ろうとしてjamiroquaiのポスター(以下)をパクってポスターの原案をマイコン部のdiscord鯖に出しました。\\
そしたら、H.Taido氏がPSのジャケット風に改造してくれました。そこで、「ランチャーもPS風にしよう」と言ったらH.Taido氏が「お前が作れ(意訳)」と言ったわけです。(実際かなり忙しいらしいです。しらんけど)\\
丁度私も麻布とのコラボ企画のゲームのためにUnityを触り始めていて、そこで私が請け負うことになりました。\\
\section{余談1}
以上の画像はPS1ですが実際できあがったランチャーはPS2のものという件について。\\
私は昔からゲームコンソールと言えば専らPS2(古い)なのですが、PS1は故障中、PS3は起動したことがないため、一番馴染みのあるPS2でやろうという話になりました。\\
しかもPS2でやるゲームがPS1のゲームなので、何ら抵抗なくPS2を選びました。(しかも、PS1は背景をパクるが難しそうだったり起動音命な所はある、という点もある)\\
\section{UIの作成}
まず、初期状態のランチャーは以下のような感じでした。\\
この時点で大分PS2らしい感じはありますが、平面っぽくなりすぎていまいちな感じはあります。そこで、ライトを移動して以下のようにしました。\\
あとは、ライト等をいじって舞台はひとまず完成です。ちなみに、キューブは本家(でPS1のゲームのセーブデータを読み込んだとき)を意識して、スケールを(x,y,z)=(10,10,5)としています。\\
次に、キューブに画像を貼りました。さらに枠足りなかったので15個に増殖。\\
さらに(画像では分かりづらいですが)解像度設定を1920*1080に変更しました。これでブラウザ画面は制作完了です。(キー操作が複雑になるので、今回あえて詳細画面は作りませんでした)

% ソースコードを挿入するときは以下のようにしてください
\begin{lstlisting}[caption=Sample]
mes "hello, world!"
\end{lstlisting}

\end{multicols*}
\end{document}
