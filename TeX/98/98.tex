% 2023本郷祭 マイコン部 部誌 「マイコンHONGOMagazine」 TeXテンプレート

% =環境設定=

\documentclass[b5paper,9pt]{jsarticle}

% 数式
\usepackage{amsmath,amsfonts}
\usepackage{bm}

% 画像
\usepackage[dvipdfmx]{graphicx}

% 段組
\usepackage{multicol}
\setlength{\columnseprule}{0.5pt}

% 余白
\usepackage[paper=b5j,truedimen,margin=15truemm]{geometry}

% ページ番号を削除
\pagestyle{empty}

% ソースコード環境
\usepackage{listings,jlisting}
\lstset{
  basicstyle={\scriptsize\ttfamily},
  identifierstyle={\scriptsize},
  commentstyle={\smallitshape},
  keywordstyle={\scriptsize\bfseries},
  ndkeywordstyle={\scriptsize},
  stringstyle={\scriptsize\ttfamily},
  frame={tb},
  breaklines=true,
  columns=[l]{fullflexible},
  numbers=left,
  xrightmargin=0zw,
  xleftmargin=0zw,
  numberstyle={\scriptsize},
  stepnumber=1,
  numbersep=1zw,
  lineskip=0.5ex
}
\renewcommand{\lstlistingname}{リスト}

% 書きたい内容に合わせてお好きなパッケージを導入していただいて構いませんが、外部パッケージは提出時に必ず合わせて提出してください。
% また、そのパッケージを使用している部分には必ずコメントをしてください。

% =環境設定ここまで=

\begin{document}

% =タイトル=
\title{PC-9800完全入門}
\author{H.Taido}
\date{\today}
\maketitle
% 初めのページのページ番号を削除(\maketitleの影響を回避)
\thispagestyle{empty}

% ここではページごとに段組しています。大きく画像を表示したいときなどは以下を \begin{multicols}{3} 、\end{multicols}のように変更すると文章が終わった直後から空白になります
\begin{multicols*}{3}
  
% 以下本文
\section[short]{ピポッ!}
皆さんこんにちは。名誉部長のH.Taidoです。


% ソースコードを挿入するときは以下のようにしてください
\begin{lstlisting}[caption=Sample]
  import { Card, Typography, Divider, Tabs, Tab } from "@mui/material";
  import Marquee from "react-fast-marquee";
  
  import ProjectList from "../../../components/ProjectList/container/ProjectList";
  import { TitleCard } from "../../../components/TitleCard/container/TitleCard";
  import ProjectHeader from "../../projectdetail/presenter/ProjectHeader";
  import { useState } from "react";
  
  export default function CurrentEvent() {
    const [place, setPlace] = useState("講堂");
  
    const onChangeTab = (event, newValue) => {
      setPlace(newValue);
    };
\end{lstlisting}

\end{multicols*}
\end{document}